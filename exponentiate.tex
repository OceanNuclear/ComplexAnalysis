%`
%\nonstopmode
\hbadness=100000
\documentclass[a4paper, 12pt]{article}
\usepackage{amsmath,amsfonts,caption,float,geometry,graphicx,mathtools,pythonhighlight,textcomp,url,verbatim,subcaption,tabularx, longtable, ulem, hyperref, tikz} %,parskip
\geometry{ a4paper, total={170mm,257mm}, left=20mm, top=20mm}
\newcommand{\matr}[1]{\underline{\underline{\textbf{#1}}}}
\newcommand{\ve}[1]{\boldsymbol{#1}}
\newcommand{\pythoncode}[2]{
\DeclareMathOperator*{\argmax}{arg\,max}
\DeclareMathOperator*{\argmin}{arg\,min}
\begin{adjustwidth}{-1.3cm}{-1.3cm}
\texttt{#1}
\inputpython{#2}{1}{1500}
\end{adjustwidth}
}
\usepackage[toc, page]{appendix}
% \usepackage[dvipsnames]{xcolor}
% \definecolor{subr}{rgb}{0.8, 0.33, 0.0}
% \definecolor{func}{rgb}{0.76, 0.6, 0.42}

\begin{document}
\centering
% \includegraphics[width=8cm]{CoverPage/UoBlogo.pdf}
% \hrule
% \bigbreak
% \textbf{F}usion Neutron \textbf{Acti}vation Spectra \textbf{U}nfolding by \textbf{N}eural \textbf{N}etworks \\
% (FACTIUNN)                                      \\
% \hrule
% \bigbreak
% \begin{minipage}[b]{0.4\textwidth}
%     \includegraphics[height=2cm]{CoverPage/CCFElogo.jpeg}
%   \end{minipage}
%   \hfill
%   \begin{minipage}[b]{0.4\textwidth}
%     \includegraphics[height=3cm]{CoverPage/UKAEAlogo.jpeg}
% \end{minipage}
    
\begin{table}[!h]
\centering
\begin{tabular}{rl}
author:&Ocean Wong          \\
       &(Hoi Yeung Wong)    \\
date:  &May 2020       \\
Organization:&Culham Centre for Fusion Energy\\
            & Sheffield Hallam University\\
Reivewed by:&Bashir Mitchell\\
Motivated by:&Boredom of quarantine
\end{tabular}
\end{table}
\hrule
\abstract
This document explores the intuition for exponentiation of complex numbers

% \hline
% \twocolumn
\begin{center}
\chapter{}
\end{center}
\section{Introduction}
To most physics students and graduates, two of the three basic arithmetic operations of complex numbers $z \in \mathbb{C}$ has been made intuitive:
\begin{longtable}{ccc}
\hline
operation& representation & intuition\\
\hline
addition & $z_1+z_2$ & translation of the $z_1$ plane by the vector $z_2$\\
multiplication & $(z_1)(z_2)$ & dilation of the complex plane by a factor $|z_2|$\\ && and rotation by a factor $arg(|z_2|)$. \\
\hline
\end{longtable}

However, the third commonly used arithmetic operation, exponentiation ${z_1}^{z_2}$, is not made obvious/intuitive as part of their physics training.


\section{Set up and calculation}
In the previous two operations we have visualized $z_1$ as a member on the infinitely extending complex plane, while $z_2$ describes an operation on this plane\footnote{This asymmetric nature of $z_1$ and $z_2$ makes the communtativity of complex multiplication unintuitive. But that's a story for another time}. We will use the same approach in the following set up:

Let $z_1= e^{x} e^{iy}$. And for convenience, let's denote the $z_2 = a+ib$.

Then an operation $z_1^{z_2}$ will give 
    
\begin{align}
z_3 = z_1^{z_2}  & = e^{ax} e^{iay} \cdot e^{ibx} e^{-by}\\
                & = e^{ax - by} e^{iay + bx} \\
                & = e^{(ax-by) +i (ay+bx)} \label{initial algebra}
\end{align}

\section{Trick}
The trick here is then to recognize the matrix-algebra nature of equation \ref{initial algebra}.

It the real part of the exponent looks like the dot product $ (a, -b) \cdot (x, y)$, while
the imaginary part of the exponent looks like the dot product $(b, a) \cdot (x, y)$.

Thus we can re-write the the real and imaginary part of the resulting $z_3$ into a 2D-vector, given by the transformation
\begin{equation}
\begin{pmatrix}
a & -b\\
b & a
\end{pmatrix}
\begin{pmatrix}
x\\
y
\end{pmatrix}\label{transformation equation}
\end{equation}

\section{Intuition}
Therefore the intuition of raising $z_1$ to the power of $z_2$ can be broken down into four steps
\begin{enumerate}
    \item decompose $z_1$ into $x=ln(|z|), y=arg(z)$. This will turn the polar coordinates lines on the argand diagram \ref{polar} into a cartesian coordinate with $-\infty < x < \infty$,$ -\pi\le y\le \pi$ (figure \ref{cartesian}). The point ($-\infty$, whatever) in figure \ref{cartesian} corresponds to the origin in figure \ref{polar}.
    \item contruct a matrix 
    $\begin{pmatrix}
    a & -b\\
    b & a
    \end{pmatrix}$ by decomposing $z_2$ into $z_2=a+ib$.
    \item apply the matrix transformation (equation \ref{transformation equation})
    \item now exponentiate (inverse transformation of what we've done in step 1).
\end{enumerate}

\begin{minipage}[b]{0.4\textwidth}
    \includegraphics[width=\textwidth]{polar.png}
    \captionof{figure}{Argand diagram with polar coordinate lines overlaid. Exponentiating figure \ref{cartesian} will give this diagram.
    The origin in figure \ref{cartesian} corresponds to the point $(1,0)$ in figure \ref{polar}.
    }\label{polar}
\end{minipage}
  \hfill
\begin{minipage}[b]{0.4\textwidth}
    \includegraphics[width=\textwidth]{cartesian.png}
    \captionof{figure}{Cartesian coordinates of the exponent of a complex number $z$. This can be obtained by taking the logarithm of figure \ref{polar} to give $x=ln(|z|), y=arg(z)$.}\label{cartesian}
\end{minipage}

\section{Correspondence to our expectation in special cases}
When $a=n, b=0$, 
$\begin{pmatrix}
a & -b\\
b & a
\end{pmatrix}$ becomes a simple scaling matrix with no skewing component, agreeing with our expectation that ${z_1}^{n}$ simply multiplies itself by n times.

If both the real and imaginary parts are zero, then the transformation in step 3 will squash everything down to the origin; and step four will transform the origin back to the number $1+0i$, confirming our expectation that $z_1^0 = 1$.

\section{Interesting results}
When $z_2$ is purely imaginary, the unit circle of $z_1$ (whose exponent has real part=0) will be mapped to the real number line in the output $z_3$. This is because when $a=0$, the matrix transformation becomes a 90-degrees rotation plus a scaling operation, swapping the real and the imaginary part of the exponent.
\section{Multiple results}
Of course, the point $(x,y)$ and the point $(x, y + 2n\pi)$ on figure \ref{cartesian} both maps to the same point on figure \ref{polar}. But these two points may be transformed differently by step 3. Therefore one get multiple results of $z_3$ depending on which represenation of $z_1$ s/he has chosen for themselves on figure \ref{cartesian} to represent $z_2$.
% \begin{figure}[H]
% \centering
% \includegraphics[width=1\textwidth]{path/to/file.png} %Stupid latex doesn't allow two dots in the filename.
% \caption{Short description} \label{somethingcatchy}
% \end{figure}

\end{document}
%`